\currfilepath
\vspace{1cm}

\tikzset{
  basic/.style  = {draw, text width=3.5cm, drop shadow, font=\sffamily, rectangle},
  root/.style   = {basic, rounded corners=2pt, thin, align=center,
                   fill=teal!25},
  level 2/.style = {basic, rounded corners=6pt, thin,align=center, fill=teal!50,
                   text width=8em},
  level 3/.style = {basic, thin, align=left, fill=green!10, text width=6.5em}
}

\begin{center}
    \begin{tikzpicture}[
        level 1/.style={sibling distance=40mm},
        edge from parent/.style={->,draw},
        >=latex]

        % root of the the initial tree, level 1
        \node[root] {E-Govornment}
        % The first level, as children of the initial tree
        child {node[level 2] (c1) {E-Voting}}
        child {node[level 2] (c2) {E-ID}};

        % The second level, relatively positioned nodes
        \begin{scope}[every node/.style={level 3}]

        \node [below of = c1, xshift=15pt] (c11) {Architektur};
        \node [below of = c11] (c12) {Sicherheit};
        \node [below of = c12] (c13) {Vorfälle};

        \node [below of = c2, xshift=15pt] (c21) {Auth};
        \node [below of = c21] (c22) {Sicherheit};
        \node [below of = c22] (c23) {Vorfälle};

        \end{scope}

        % lines from each level 1 node to every one of its "children"
        \foreach \value in {1,2,3}
        \draw[->] (c1.195) |- (c1\value.west);

        \foreach \value in {1,...,3}
        \draw[->] (c2.195) |- (c2\value.west);

        \end{tikzpicture}
 \end{center}